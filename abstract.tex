The joint observation of GW170817 in the electromagnetic (EM) regime and using
gravitational waves (GW) ushered in the era of multimessenger astronomy. Fervent follow
up observations just after the GW trigger time revealed the properties of the associated
SGRB jet and optical kilonova in exquisite detail. Since then, the AdvLIGO and VIRGO
detectors have detected multiple events in the GW regime, most of which are binary black
hole (BBH) mergers. However, it is known that some of these events are neutron star (NS)
compact binary coalescences (CBC), where in at least one of the components is a neutron
star.\\
These systems produce much richer EM outflows when compared to BBH systems, since the
baryonic matter in the NS can be tidally disrupted to produce a litany of associated
features across the EM spectrum. These features can help shed light on the behaviour of
matter at supranuclear densities and extreme gravity, and also contain information about
the long elusive neutron star Equation of State (EoS).\\
However, GW170817 has remained thus far the only event which has launched strong signals
both in the EM and GW windows. Other events have been detected since using GW, although
their EM counterparts have been limited to weak gamma-ray jets, presumably due to being
observed off-axis. Due to this reason, conclusions about the properties of such EM
counterparts are uncertain. Additionally, like GW170817 which is believed to be a binary
neutron star (BNS) merger, neutron star-black hole (NSBH) mergers also can launch jets
and kilonovae given the right binary component parameters, as shown using numerical
relativity simulations. However, in BNS mergers, the remnant quickly collapses into a
black hole which can affect the brightness of the EM counterparts such as the
kilonova.\\
In this report, a framework is developed for the computation of the EM counterparts of
NS mergers, specifically focusing on the outflows from NSBH mergers. This framework
computes the GW network SNR for a given NSBH merger, and also computes the remnant mass
left outside the remnant BH's apparent horizon, the dynamically ejected mass and the
mass of the accretion disc (if one is formed) around the remnant BH. The latter mass
components all play an important role in deciding the detectability of the EM
counterparts. Thus this framework aims to simulate NSBH mergers both in the GW and EM
windows.
