\chapter{Results and Future Work}\label{ch:results-future}

    In this report, a framework was developed to compute the EM counterparts produced
    when an NSBH binary with a given set of parameters merges. This framework takes as
    inputs:

    \begin{itemize}

        \item The set of binary parameters, which are usually samples from a population
            model distribution for each of the binary parameters. Some of these
            population models are physically motivated and some are empirical guesses
            in situations where a physical model has many complications to consider.

        \item The functions used to compute the counterpart energetics and structure. In
            the case of the SGRB jet, the former are the functions given in
            \cite{kawaguchi_2016} and \cite{foucart_2018} to compute the dynamic and
            remnant masses, whereas the latter is the Gaussian structured jet model from
            \cite{salafia_2015}, \cite{saleem_2020B}.

    \end{itemize}

    and simulates the EM counterparts using just these equations. Additionally, these
    events are also analysed from the GW side of things, with the corresponding GW
    optimal SNR also being computed under this framework.\\
    From these simulations, it is seen that under the assumption that NSBH mergers are
    disruptive (which may be a strong assumption to make, given the mass ratio
    distributions seen by Advanced LIGO/VIRGO), in order to have bright SGRB jets the
    binary must have a low mass black hole that is spinning rapidly. This conclusion is
    seen from considering the effect of the BH spin distribution on the number of EM
    detected events in \ref{tab:popln_numbers}. Furthermore this also implies that from
    the observed number of SGRB jets, one can infer the nature of the BH spin
    distribution in an NSBH merger, and compare it to more physically motivated
    distributions, say those derived from stellar population synthesis codes.\\
    As for the events from GWTC-1 and 2 which were considered here, it is seen that for
    all these events, namely GW170817, GW190425 and GW 190426\_152155, which were
    observed in the GW regime, there is a non-negligible possibility that the event was
    an NSBH event which produced an SGRB jet (along with the other EM counterparts, in
    the case of GW170817). Only in the case of GW170817 and GW190425 were the jets
    actually observed as detections, although all their apparent isotropic equivalent
    energies are relatively lower when compared to the typical cosmological population
    This observation is explained if the events were observed off-axis, although only
    for GW170817 has afterglow modelling constrained the off-axis viewing angle
    well enough.\\
    With these conclusions in mind, the directions in which this work will be taken in
    the future is explored in the next section. These directions are directly related to
    loosening the assumptions made in this work in order to simplify the problem and
    serve as a reminder to the applicability of the work.

    \section{Future Directions}

        \begin{enumerate}

            \item In this work, the spin of the black hole is assumed to be aligned to
                the angular momentum of the binary system, and also it is assumed that
                the system is non-precessing. The former assumption is valid to make for
                the computation of the masses left outside the BH apparent horizon,
                since retrograde spins disfavour disruption nonetheless. However, the BH
                spins which are tilted with respect to the angular momentum of the
                binary can still disrupt inspiralling NS matter (although to a lesser
                amount that aligned spin) and is an assumption that must be relaxed as
                much as possible. The relaxation of the latter assumption will introduce
                complications, since for a precessing binary system the definition of
                the inclination angle changes with time, and so the entire framework
                will have to be revamped in order to take care of that new definition.

            \item Here, the microphysics behind the NS Equation of State (EoS) and the
                resulting effect it can have on the EM counterparts is not considered.
                As was mentioned, it is known qualitatively that `softer' equations of
                state can disfavour disruption whereas the `harder' equations of state
                aid disruption, and thus would support more energetic jets. For the
                purposes of computation within the framework, the NS EoS is assumed to
                be the SFHo EoS (see \cite{hempel_2010}, \cite{hempel_2012}) which
                predicts a NS radius of $\sim 11$ km and a tidal deformability of around
                330 for the NS mass of $1.4$ M$_\odot$, which is the median mass for
                GW170817 considered as a BNS merger.  However, from the posterior
                distributions on tidal deformability for GW170817 in combination with
                the constraints on the same from EM observations of AT2017gfo, a wide
                range of EoS can explain the observed properties. Thus, this is a vital
                area of the parameter space that must be explored more deeply.

            \item The framework in its current state, only computes the properties of
                a possible SGRB jet from an NSBH merger using the aformentioned fit
                formulae. However, as seen from \ref{fig:nsbh_outflows}, there are other
                outflows as well which arise from NSBH mergers such as the optical
                kilonova and the jet afterglow. These are non-relativistic outflows and
                are thus not affected as drastically by off-axis viewing, unlike the
                SGRB jet. Furthermore, for black holes in the `mass gap' region (i.e.
                $M_{BH} \sim (3, 5)$ M$_\odot$), the brightness of the kilonova may even
                be used to distinguish between NSBH merger events and BNS merger events
                (see \cite{barbieri_2019b}). Thus, this component of the EM outflows is
                another vital piece of information that is being modelled, and will be
                investigated in the future. However, the jet afterglow component is
                sensitive to the properties of the environment surrounding the NS
                binary, and thus thus deriving general conclusions from considering this
                component requires more careful analysis or simplifying assumptions.

            \item It is also assumed currently that the only mechanism for the
                extraction of the SGRB jet is via the Blandford-Znajek mechanism, which
                kicks in once the disruption of the inspiralling NS occurs and the
                remnant matter accretes around the remnant BH. However, this may not be
                the case, and alternative mechanisms have been proposed which do not
                rely on tidal disruption of the NS, such as the mechanism in
                \cite{east_2020}. Such alternative mechanisms, though exotic, may
                explain why EM outflows from NSBH mergers have not been distinctly
                observed yet.

            \item Once the other assumptions are relaxed, further analysis can be done
                to infer the jet and kilonova structure parameters. This can be done by
                using GWBENCH or a similar tool which computes the FIM for a particular
                set of GW network and source parameters. The FIM can then be used to
                forecase the parameter estimates for the component masses, spins,
                luminosity distances etc., which will then translate into fluences (for
                the associated prompt emission) and magnitudes (for the associated
                kilonova) using the underlying fit formulae for each component. Thus, by
                comparing these observed quantitites with what is predicted, one should
                be able to constrain the structure parameters. However, note that this
                comes with the caveat that the corresponding NSBH merger must have a
                high enough SNR ($\gtrsim 20$) for the analysis in \ref{sec:ns_in_gw} to
                be valid. For events with only a moderately high SNR (but still above
                the SNR detection threshold), traditional Bayesian parameter inference
                must be carried out to compute the posteriors accurately.

        \end{enumerate}
