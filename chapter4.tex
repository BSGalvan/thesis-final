\chapter{NS Merger Candidates in GWTC-2}\label{ch:candidates}

\section{About GWTC-2}

    The first half of the third observing run (O3a) of the LVC started on the 1st of
    April 2019, and went on till 30th of September, 2019.  Following this, instrumental
    upgrades were made during the month of October and the second half of the third
    observing run (O3b) was started on the 1st of November. Due to the global pandemic,
    the observing run had to be prematurely suspended on 30th of March, 2020.\\ Starting
    with O3, alerts were distributed via the public alerts section of Gravitational-wave
    Candidate Event Database (GRACE-DB). If triggers were registered that passed the
    detection threshold of the LVC network during observational runs, online parameter
    estimation was done using the raw GW data, and low latency estimates of the rough
    sky position, component masses and luminosity distance was made available for
    observers in the EM regime. This allowed for rapid follow-up in various bands of the
    electromagnetic spectrum, and these observations were reported and cross-verified
    using NASA's GRB Coordinates Network (GCN). Using the circulars reported in the GCN
    for NSBH/BNS events of interest, along with the low-latency information from
    GRACE-DB, O3a's non-BBH candidate events have been collected in table \ref{tab:gcns}
    \\ More detailed analysis of these events in the months following has led to at
    least 53 events in the third observing run alone. These can be classified as:
    \begin{itemize}

        \item 37 Binary Black Hole (BBH) merger candidates.

        \item 7 BNS merger candidates. Of these, only 1 corresponds to O3a, which is the
            event GW190425 discussed in the chapter \textbf{sec:190425}.

        \item 4 events in the mass gap, which are events with compact objects with
            masses of 3-5 $M_{\odot}$.

        \item 5 NSBH merger candidates. Of these, only 1 has been confirmed officially,
            which is the event GW190814.

    \end{itemize}

    Of these, 26 events were officially confirmed and 13 new events were reported for
    the first time in \cite{abbott_2020}, and it is from there that the posterior
    distributions of the various parameters (such as inclination angle $\iota$,
    luminosity distance $D_L$ etc) are used for further analysis.  Note also that there
    are several marginal events that have been reported, in that they have
    non-negligible probability distributed between two classifications. For example, the
    event GW190426\_152155 has significant probability split between it being a
    terrestrial event (58 \%) and a NSBH/BNS/Mass Gap event (cumulatively 42\%). Without
    a significant EM counterpart, this event cannot be confidently placed in either of
    the classes, and thus warrants further analysis. This is the subject of chapter
    \textbf{sec:190426}.

    \begin{table}[H]  % now this is how you format a table ;)
        \centering
        \setlength{\extrarowheight}{0pt}
        \addtolength{\extrarowheight}{\aboverulesep}
        \addtolength{\extrarowheight}{\belowrulesep}
        \setlength{\aboverulesep}{0pt}
        \setlength{\belowrulesep}{0pt}
        \begin{tabular}{cccccc}
            \toprule
            \rowcolor[rgb]{0.71,0.851,1}
            \begin{tabular}[c]{@{}>{\cellcolor[rgb]{0.71,0.851,1}}c@{}}
                \textbf{GRACE-DB} \\
                \textbf{Superevent} \\
                \textbf{ID}
            \end{tabular} &
            \textbf{$\mathcal{P}$(BBH)} & \textbf{$\mathcal{P}$(BNS)} &
            \textbf{$\mathcal{P}$(MassGap)} & \textbf{$\mathcal{P}$(NSBH)} &
            \textbf{$\mathcal{P}$(Terr)}  \\
            \hline \uline{S190426c} & 0 & 24 & 12 & 6 & 58\\
            S190910h & 0 & 61 & 0 & 0 & 39 \\
            S200213t & 0 & 63 & 0 & 0 & 37 \\
            S191213g & 0 & 77 & 0 & 0 & 23 \\
            S190901ap & 0 & 86 & 0 & 0 & 14 \\
            \uline{S190425z} & 0 & \textbf{\textgreater{}99} & 0 & 0 & 0 \\
            S190930s & 0 & 0 & 95 & 0 & 5 \\
            S190923y & 0 & 0 & 0 & 68 & 32 \\
            S190930t & 0 & 0 & 0 & 74 & 26 \\
            S191205ah & 0 & 0 & 0 & 93 & 7 \\
            S190910d & 0 & 0 & 0 & 98 & 2 \\
            S190814bv & 0 & 0 & 0 & \textbf{\textgreater{}99} & 0 \\
            \bottomrule
        \end{tabular}
        \caption[Candidate Merger Events and GRACE-DB Superevent IDs]{
                List of candidate merger events and the probability of
                classification (reported in \%) for each non-BBH event reported
                during O3a. Here the GRACE-DB superevent ID is used (instead of the
                GWTC-2 event ID), since for a particular GW event, the superevent
                collects both the EM followup as well as GW trigger information
                within GRACE-DB. The probabilities are assigned using the process
                described in \cite{kapadia_2020}, and are reported
                in GRACE-DB. The events underlined are discussed in more detail in
                later chapters.
            }
    \end{table}


    \begin{table}[H]
        \centering
        \setlength{\extrarowheight}{0pt}
        \addtolength{\extrarowheight}{\aboverulesep}
        \addtolength{\extrarowheight}{\belowrulesep}
        \setlength{\aboverulesep}{0pt}
        \setlength{\belowrulesep}{0pt}
        \begin{tabular}{cccc}
            \toprule
            \rowcolor[rgb]{0.71,0.851,1}
            \textbf{UID} & \textbf{FAR} & \textbf{$D_L$ (Mpc)} &
            \begin{tabular}[c]{@{}>{\cellcolor[rgb]{0.71,0.851,1}}c@{}}
                \textbf{Error in $D_L$} \\
                \textbf{(Mpc)}
            \end{tabular}  \\
            \hline
            \uline{S190426c} & 1 per 1.6276 yr & 377 & 100 \\
            S190910h & 1.1312 per yr & 230 & 88 \\
            S200213t & 1 per 1.7934 yr & 201 & 80 \\
            S191213g & 1.1197 per yr & 201 & 81 \\
            S190901ap & 1 per 4.5093 yr & 241 & 79 \\
            \uline{S190425z} & 1 per 69834 yr & 158 & 43 \\
            S190930s & 1 per 10.534 yr & 709 & 191 \\
            S190923y & 1.5094 per yr & 438 & 133 \\
            S190930t & 1 per 2.0536 yr & 108 & 38 \\
            S191205ah & 1 per 2.5383 years & 385 & 164 \\
            S190910d & 1 per 8.5248 years & 606 & 197 \\
            S190814bv & 1 per 1.559e+25 years & 241 & 26 \\
            \bottomrule
        \end{tabular}
        \caption[Candidate Merger Events and FARs]{
                List of candidate merger events and the probability of
                classification for each non-BBH event reported during O3a. FAR
                refers to the False Alarm Rate (in number of events per year or
                equivalently in Hz$^{-1}$), and $D_L$ is the luminosity distance.
                Both these values are those reported in GRACE-DB corresponding to
                each event. The events underlined are discussed in more detail in
                later chapters.
            }
    \end{table}

    \begin{table}[H]
        \centering
        \setlength{\extrarowheight}{0pt}
        \addtolength{\extrarowheight}{\aboverulesep}
        \addtolength{\extrarowheight}{\belowrulesep}
        \setlength{\aboverulesep}{0pt}
        \setlength{\belowrulesep}{0pt}
        \begin{tabular}{ccccc}
            \toprule
            \rowcolor[rgb]{0.71,0.851,1}
            \textbf{UID} &
            \textbf{FERMI-LAT} &
            \textbf{FERMI-GBM} &
            \textbf{SWIFT/BAT} &
            \textbf{INTEGRAL} \\
            \hline
            \uline{S190426c} & 24342 & 24248 & 24255 & 24242\\
            S190910h & 25742 & 25714 & 25718 & 25709 \\
            S200213t &
                27062 &
                {\cellcolor[rgb]{1,0.749,0.749}} \textcolor{red}{27056} &
                {\cellcolor[rgb]{1,0.749,0.749}} \textcolor{red}{27058} &
                27050 \\
            S191213g & 26412 & 26409 & 26410 & 26401 \\
            S190901ap & 25625 & 25610 & 25617 & 25605 \\
            \uline{S190425z} &
                24266 &
                24185 &
                24184, 24296 &
                \begin{tabular}{c}
                    24169, 24170 \\
                    24178, 24181
                \end{tabular} \\
            S190930s & 25895 & 25886 & 25889 & 25872 \\
            S190923y & 25834 & 25823 & 25846 & 25815, 25825 \\
            S190930t & 25898 & 25887 & 25888 & 25880 \\
            S191205ah & 26363 & 26359 & 26365 & 26531 \\
            S190910d & 25717 & 25699 & 25704 & 25698 \\
            S190814bv & 25385 & 25326 & 25341 & 25323 \\
            \bottomrule
        \end{tabular}
        \caption[Candidate merger events and relevant GCNs]
            {
                List of candidate merger events and the probability of
                classification for each non-BBH event reported during O3a. Here the
                GCN Circular number reporting the findings of the particular
                instrument in the column heading is reported, corresponding to each
                event. The events underlined are discussed in more detail in later
                chapters. GCNs marked with red should be ignored during analysis
                since they correspond to times when the respective instruments were
                in the South Atlantic Anomaly (SAA).
            }
        \label{tab:gcns}
    \end{table}

\section{Summary}
