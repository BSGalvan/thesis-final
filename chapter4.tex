\chapter{NS Merger Candidates in GWTC-2}\label{ch:candidates}

\section{About GWTC-2}

    The first half of the third observing run (O3a) of the LVC started on the 1st of
    April 2019, and went on till 30th of September, 2019.  Following this, instrumental
    upgrades were made during the month of October and the second half of the third
    observing run (O3b) was started on the 1st of November. Due to the global pandemic,
    the observing run had to be prematurely suspended on 30th of March, 2020.\\
    Starting with O3, alerts were distributed via the public alerts section of
    Gravitational-wave Candidate Event Database (GRACE-DB). If triggers were registered
    that passed the detection threshold of the LVC network during observational runs,
    online parameter estimation was done using the raw GW data, and low latency
    estimates of the rough sky position, component masses and luminosity distance was
    made available for observers in the EM regime. This allowed for rapid follow-up in
    various bands of the electromagnetic spectrum, and these observations were reported
    and cross-verified using NASA's GRB Coordinates Network (GCN). Using the circulars
    reported in the GCN for NSBH/BNS events of interest, along with the low-latency
    information from GRACE-DB, O3a's non-BBH candidate events have been collected in
    table \ref{tab:gcns}.\\
    More detailed analysis of these events in the months following has led to at least
    53 events in the third observing run alone. These can be classified as:
    \begin{itemize}

        \item 37 Binary Black Hole (BBH) merger candidates.

        \item 7 BNS merger candidates. Of these, only 1 corresponds to O3a, which is the
            event GW190425 discussed in the chapter \ref{sec:190425}.

        \item 4 events in the mass gap, which are events with compact objects with
            masses of 3-5 $M_{\odot}$.

        \item 5 NSBH merger candidates. Of these, only 1 has been confirmed officially,
            which is the event GW190814.

    \end{itemize}

    Of these, 26 events were officially confirmed and 13 new events were reported for
    the first time in \cite{abbott_2020A}, and it is from there that the posterior
    distributions of the various parameters (such as inclination angle $\iota$,
    luminosity distance $D_L$ etc) are used for further analysis.  Note also that there
    are several marginal events that have been reported, in that they have
    non-negligible probability distributed between two classifications. For example, the
    event GW190426\_152155 has significant probability split between it being a
    terrestrial event (58 \%) and a NSBH/BNS/Mass Gap event (cumulatively 42\%). Without
    a significant EM counterpart, this event cannot be confidently placed in either of
    the classes, and thus warrants further analysis. This is the subject of chapter
    \ref{sec:190426}.

    \begin{table}[H]  % now this is how you format a table ;)
        \centering
        \setlength{\extrarowheight}{0pt}
        \addtolength{\extrarowheight}{\aboverulesep}
        \addtolength{\extrarowheight}{\belowrulesep}
        \setlength{\aboverulesep}{0pt}
        \setlength{\belowrulesep}{0pt}
        \begin{tabular}{cccccc}
            \toprule
            \rowcolor[rgb]{0.71,0.851,1}
            \begin{tabular}[c]{@{}>{\cellcolor[rgb]{0.71,0.851,1}}c@{}}
                \textbf{GRACE-DB} \\
                \textbf{Superevent} \\
                \textbf{ID}
            \end{tabular} &
            \textbf{$\mathcal{P}$(BBH)} & \textbf{$\mathcal{P}$(BNS)} &
            \textbf{$\mathcal{P}$(MassGap)} & \textbf{$\mathcal{P}$(NSBH)} &
            \textbf{$\mathcal{P}$(Terr)}  \\
            \hline \uline{S190426c} & 0 & 24 & 12 & 6 & 58\\
            S190910h & 0 & 61 & 0 & 0 & 39 \\
            S200213t & 0 & 63 & 0 & 0 & 37 \\
            S191213g & 0 & 77 & 0 & 0 & 23 \\
            S190901ap & 0 & 86 & 0 & 0 & 14 \\
            \uline{S190425z} & 0 & \textbf{\textgreater{}99} & 0 & 0 & 0 \\
            S190930s & 0 & 0 & 95 & 0 & 5 \\
            S190923y & 0 & 0 & 0 & 68 & 32 \\
            S190930t & 0 & 0 & 0 & 74 & 26 \\
            S191205ah & 0 & 0 & 0 & 93 & 7 \\
            S190910d & 0 & 0 & 0 & 98 & 2 \\
            S190814bv & 0 & 0 & 0 & \textbf{\textgreater{}99} & 0 \\
            \bottomrule
        \end{tabular}
        \caption[Candidate Merger Events and GRACE-DB Superevent IDs]{
                List of candidate merger events and the probability of
                classification (reported in \%) for each non-BBH event reported
                during O3a. Here the GRACE-DB superevent ID is used (instead of the
                GWTC-2 event ID), since for a particular GW event, the superevent
                collects both the EM followup as well as GW trigger information
                within GRACE-DB. The probabilities are assigned using the process
                described in \cite{kapadia_2020}, and are reported
                in GRACE-DB. The events underlined are discussed in more detail in
                later chapters.
            }
    \end{table}


    \begin{table}[H]
        \centering
        \setlength{\extrarowheight}{0pt}
        \addtolength{\extrarowheight}{\aboverulesep}
        \addtolength{\extrarowheight}{\belowrulesep}
        \setlength{\aboverulesep}{0pt}
        \setlength{\belowrulesep}{0pt}
        \begin{tabular}{cccc}
            \toprule
            \rowcolor[rgb]{0.71,0.851,1}
            \textbf{UID} & \textbf{FAR} & \textbf{$D_L$ (Mpc)} &
            \begin{tabular}[c]{@{}>{\cellcolor[rgb]{0.71,0.851,1}}c@{}}
                \textbf{Error in $D_L$} \\
                \textbf{(Mpc)}
            \end{tabular}  \\
            \hline
            \uline{S190426c} & 1 per 1.6276 yr & 377 & 100 \\
            S190910h & 1.1312 per yr & 230 & 88 \\
            S200213t & 1 per 1.7934 yr & 201 & 80 \\
            S191213g & 1.1197 per yr & 201 & 81 \\
            S190901ap & 1 per 4.5093 yr & 241 & 79 \\
            \uline{S190425z} & 1 per 69834 yr & 158 & 43 \\
            S190930s & 1 per 10.534 yr & 709 & 191 \\
            S190923y & 1.5094 per yr & 438 & 133 \\
            S190930t & 1 per 2.0536 yr & 108 & 38 \\
            S191205ah & 1 per 2.5383 years & 385 & 164 \\
            S190910d & 1 per 8.5248 years & 606 & 197 \\
            S190814bv & 1 per 1.559e+25 years & 241 & 26 \\
            \bottomrule
        \end{tabular}
        \caption[Candidate Merger Events and FARs]{
                List of candidate merger events and the probability of
                classification for each non-BBH event reported during O3a. FAR
                refers to the False Alarm Rate (in number of events per year or
                equivalently in Hz$^{-1}$), and $D_L$ is the luminosity distance.
                Both these values are those reported in GRACE-DB corresponding to
                each event. The events underlined are discussed in more detail in
                later chapters.
            }
    \end{table}

    \begin{table}[H]
        \centering
        \setlength{\extrarowheight}{0pt}
        \addtolength{\extrarowheight}{\aboverulesep}
        \addtolength{\extrarowheight}{\belowrulesep}
        \setlength{\aboverulesep}{0pt}
        \setlength{\belowrulesep}{0pt}
        \begin{tabular}{ccccc}
            \toprule
            \rowcolor[rgb]{0.71,0.851,1}
            \textbf{UID} &
            \textbf{FERMI-LAT} &
            \textbf{FERMI-GBM} &
            \textbf{SWIFT/BAT} &
            \textbf{INTEGRAL} \\
            \hline
            \uline{S190426c} & 24342 & 24248 & 24255 & 24242\\
            S190910h & 25742 & 25714 & 25718 & 25709 \\
            S200213t &
                27062 &
                {\cellcolor[rgb]{1,0.749,0.749}} \textcolor{red}{27056} &
                {\cellcolor[rgb]{1,0.749,0.749}} \textcolor{red}{27058} &
                27050 \\
            S191213g & 26412 & 26409 & 26410 & 26401 \\
            S190901ap & 25625 & 25610 & 25617 & 25605 \\
            \uline{S190425z} &
                24266 &
                24185 &
                24184, 24296 &
                \begin{tabular}{c}
                    24169, 24170 \\
                    24178, 24181
                \end{tabular} \\
            S190930s & 25895 & 25886 & 25889 & 25872 \\
            S190923y & 25834 & 25823 & 25846 & 25815, 25825 \\
            S190930t & 25898 & 25887 & 25888 & 25880 \\
            S191205ah & 26363 & 26359 & 26365 & 26531 \\
            S190910d & 25717 & 25699 & 25704 & 25698 \\
            S190814bv & 25385 & 25326 & 25341 & 25323 \\
            \bottomrule
        \end{tabular}
        \caption[Candidate merger events and relevant GCNs]
            {
                List of candidate merger events and the probability of
                classification for each non-BBH event reported during O3a. Here the
                GCN Circular number reporting the findings of the particular
                instrument in the column heading is reported, corresponding to each
                event. The events underlined are discussed in more detail in later
                chapters. GCNs marked with red should be ignored during analysis
                since they correspond to times when the respective instruments were
                in the South Atlantic Anomaly (SAA).
            }
        \label{tab:gcns}
    \end{table}

\section{Analysis of GW190425}\label{sec:190425}

    GW190425 (or S190425z in GRACE-DB) is a GW trigger which was recorded by the LVC on
    25th April, 2019 at 08:18 UTC. At the time of this trigger, the Hanford site of LIGO
    (H1) was undergoing maintenance, whereas the Livingston site (L1) of LIGO and the
    VIRGO detector were both operational. However, at the VIRGO detector this event was
    sub-threshold, effectively making this event a single-detector trigger. As a
    consequence, the LVC sky localisation area was wider than as compared to GW170817,
    and EM follow-up was constrained to serendipitious observations by satellites which
    happened to be observing the same area of the sky, or to diminished coverage by
    satellites due to observational schedules.

    Initial work was carried out to reproduce the results of \cite{saleem_2020B}, where
    the authors use a frequentist approach to discuss the possibility of a relativistic
    jet from the binary neutron star (BNS) merger event GW190425\footnote{
        At the time of the writing of the paper, this event was still not confirmed as a
        bona-fide GW event, and so it was denoted by GRACE-DB as S190425z. Consequently,
        several key pieces of information (such as the luminosity distance and
        inclination angle posteriors) from the GW inference was not yet public, and had
        to be worked around. See text for more details.
    }. In \cite{saleem_2020B}, three key ideas  are developed which are described in
    detail in the following sections.

    \subsection{Constraints on the \texorpdfstring{$D_L-\iota$}{dL-iota} Posterior}
    \label{sec:dl-iota_posterior}

    At the time of writing of \cite{saleem_2020B}, only the low-latency information was
    made public. Consequently, information about the inclination angle of this event was
    not released. However, both the luminosity distance ($D_L$) and inclination angle
    ($\iota$) of the event are required for the electromagnetic analysis of the
    event\footnote{
        The former is used to calculate the fluence, and the latter is directly related
        to the viewing angle, which in turn decides the observed isotropic equivalent
        energy (see Eq. \ref{eq:6})
    }. To solve this issue, a known correlation between the luminosity distance $D_L$
    and $\iota$ (see \cite{schutz_2011}, \cite{seto_2015}) was used to infer the
    distribution of $\iota$ for this event, which was possible since the former was
    publicly known.\\
    The following are the other publicly released information relevant to the problem,
    and can be used to constrain the $D_L-\iota$ joint distribution:
    \begin{itemize}

        \item The posterior probability of the event being a BNS merger is $>$ 99 \%,

        \item The event was observed by the LIGO Livingston (L1), and Virgo (V1)
            detectors, whereas the LIGO Hanford (H1) detector was not observing.
            However, at V1 the signal-to-noise ratio (SNR) was below the threshold and
            thus this event was a single detector trigger.

        \item The preliminary luminosity distance estimate is given by $D_L$ = 155 $\pm$
            41 Mpc.

    \end{itemize}

    Using these inputs the $D_L-\iota$ space is constrained as follows.
    \begin{enumerate}

        \item A population of BNS mergers is simulated, such that they are uniformly
            distributed in the comoving volume, with the inclination angle of the
            binaries being such that $\cos \iota \in [-1, 1]$. This also means that the
            luminosity distance is initially distributed such that $\mathcal{P}(D)
            \propto D^2$ upto some threshold distance. For the purposes of the
            simulations, this threshold distance is set to be the distance corresponding
            to the 99\% percentile of a Gaussian with a mean of 155 Mpc and a standard
            deviation of 41 Mpc. In practice, it is the maximum distance up till which
            the comoving and the luminosity distances can be used interchangeably, which
            corresponds to a redshift of roughly 0.1.

        \item The NS masses are uniformly distributed between 1-2 $M_{\odot}$. This
            enforces the constraint that from the analysis of the GW waveform, the event
            has > 99 \% probability of being a BNS merger.

        \item Then, the optimal SNR is computed for each realisation using the
            restricted post-Newtonian (PN) waveform (RWF) (see \cite{cutler_1994},
            \cite{kastha_2020}). Usually, the process of matched filtering is
            done to detect GW signals amongst background noise, where various template
            waveforms are cross-correlated with the observed data. These templates all
            correspond to mergers with various signal parameters such as masses, spins
            etc., and the template which maximizes the SNR is the optimal template, with
            the corresponding SNR being the optimal SNR, defined as :
            \begin{equation}
                \rho = \sqrt{4 \int_0^\infty \dfrac{|\tilde{h}(f)|^2}{S_h(f)} df}
            \end{equation}
            where $S_h(f)$ is the detector's power spectral density (PSD) and
            $\tilde{h}(f)$ is the frequency domain gravitational waveform.  For the RWF,
            this is given as $\tilde{h}(f) = \mathcal{A} f^{-7/6} e^{i \psi(f)}$, where
            $\mathcal{A}$ is the amplitude and $\psi(f)$ is the frequency domain GW
            phase.  In this scheme, the PN corrections to the amplitude of the waveform
            are ignored but the phase is accurately accounted for, since GW parameter
            estimations are most sensitive to the phase of the waveform
            (\cite{poisson_1995}). Using the RWF, the optimal SNR for compact binary
            coalescences can be written as :
            \begin{equation}
                \rho(m_1, m_2, D_L, \theta, \phi, \psi, \iota) =
                    \sqrt{
                        4 \dfrac{\mathcal{A}^2}{D_L^2}
                        \left[
                                 F_{+}^2(\theta, \phi, \psi)(1 + \cos^2\iota)^2 +
                                 4F_{\times}^2(\theta, \phi, \psi)\cos^2\iota
                        \right]
                        I(M)
                    }
            \end{equation}
            where $F_{+, \times}(\theta, \phi, \psi)$ are the antenna pattern functions
            for the 'plus' and 'cross' polarisations, $\mathcal{A} = \sqrt{5/96}
            \pi^{-2/3} \mathcal{M}^{5/6}$. Here, $\mathcal{M}$ is the chirp mass, which
            is related to the total mass M by $\mathcal{M} = M \eta^{3/5}$, where $\eta
            = \frac{m_1 m_2}{M^2}$ is the symmetric mass ratio of the system and $m_1,
            m_2$ are the component masses. The four angles $(\theta, \phi, \psi, \iota)$
            describe the location and orientation of the source with respect to the
            detector. $I(M)$ is the frequency integral defined as :
            \begin{equation}
                I(M) =
                    \int_0^\infty \dfrac{f^{-7/3}}{S_h(f)} df \approx
                    \int_{f_{low}}^{f_{LSO}} \dfrac{f^{-7/3}}{S_h(f)}
            \end{equation}
            where, $f_{low}$ is the lower seismic cut-off for the detectors and $f_{LSO}
            = \dfrac{1}{6^{3/2}\pi M}$, which is the GW frequency at the last stable
            orbit for a BBH merger with total mass M.  To compute the optimal SNR in L1
            and V1, the best reported O2 sensitivities were used as conservative O3
            sensitivities, as an input for $S_h(f)$.  As the trigger is an L1
            single-detector trigger, the conditions that SNR < 4 at V1 and the network
            SNR > 9 are enforced. The former is motivated by the single-detector
            threshold of the GstLAL pipeline, whereas the latter is motivated by the
            fact that the network SNR of all O1/O2 events were > 9.

        \item From the resulting population, a sub-population is extracted such that the
            luminosity distance follows a Gaussian distribution with a mean of 155 Mpc
            and a standard deviation of 41 Mpc. This is done so as to impose the
            constraint applied by the luminosity distance posterior distribution
            released by the LVC.

    \end{enumerate}

        The resulting 2D distribution of $D_L-\iota$ of this sub-population is shown in
        the figure below. This is used further on, as the prior for studying the
        possibility of a sGRB from GW190425.

    \begin{figure}[H]
        \centering
        \def\svgwidth{0.9\textwidth}
        \input{figures/saleem+Fig1.pdf_tex}
        \caption[$D_L-\iota$ Posterior distribution.]{
                    Constraints on the $D_L-\iota$ joint distribution, obtained from
                    imposing the observed properties of S190425z/GW190425.
        }
        \label{fig:dl_iota}
    \end{figure}

    \subsection{Calculation of the Apparent structure}

    Assuming the intrinsic jet structure models described before (in Sec.
    \ref{sec:bns}), the apparent isotropic equivalent energy is calculated using the
    equation (as in \cite{salafia_2015} and
    \cite{biscoveanu_2020}):

     \begin{equation}
         \label{eq:6}
         E_{iso}(\theta_v) =
             \dfrac{1}{2\pi}
             \int_{0}^{2\pi} d\phi \int_{0}^{\theta_{max}} d\theta
             \sin \theta
             \dfrac{\epsilon(\theta)}{\Gamma(\theta)^4
             [1 - \beta(\theta) \cos \alpha_v]^3}
     \end{equation}

    where:
    \begin{itemize}

        \item $\theta_v$ is the viewing angle of the observer.

        \item $\epsilon(\theta)$ is the normalised energy profile function.

        \item $\alpha_v$ is the angle between the line of sight and the direction to the
            jet element at $(\theta, \phi)$, given by $\cos \alpha_v = \cos \theta_v
            \cos \theta + \sin \theta_v \sin \theta \cos \phi$.

        \item $\theta_{max}$ is the upper cut-off of polar integration\footnote{
                Such a cutoff can occur because the edge of the jet has been reached, or
                that the gamma-ray emission efficiency is lowered below a threshold, and
                so emission is negligible beyond $\theta_{max}$.
            }.

    \end{itemize}

    Thus, depending on the underlying intrinsic structure assumed (be it Gaussian or
    tophat jet), using Eq. \ref{eq:6} one can infer the apparent structure of the jet.
    Note that this is the variation of the (apparent) isotropic equivalent energy
    ($E_{iso}$) at various viewing angles ($\theta_v$), and hence gives how the sGRB
    jet, if launched, would appear to an observer at an angle to the jet's axis. This
    variation is shown in Fig. \ref{fig:e_iso}, and is used a prior for further
    analysis.

    \begin{figure}
        \centering
        \def\svgwidth{\textwidth}
        \input{figures/e_iso.pdf_tex}
        \caption[Variation of the apparent isotropic equivalent energy,
                 for observers at different viewing angles]{
                 Variation of the apparent isotropic equivalent energy, for observers at
                 different viewing angles. The figure shows both the top-hat (dark red)
                 and Gaussian (dark green) jet structures, with $E_{tot., \gamma} =
                 10^{49}$ ergs. $\theta_j$ for the top-hat jet and $\theta_c$ for the
                 Gaussian jet are both $5^{\circ}$ (marked with vertical solid bright
                 red line), and $\Gamma_0$ in both cases is 100. The horizontal dashed
                 black line denotes $E_{iso}(0)$. The orange, dash-dotted line and the
                 blue dotted lines are the tophat and Gaussian intrinsic jet structures,
                 represented as $4\pi\epsilon(\theta_v)$. For the solid green curve, the
                 entire jet emits gamma-rays, whereas for the dashed and dashed-dotted
                 green curves the emission is restricted to regions where $\Gamma \beta
                 > 15$ and $\Gamma \beta  > 30$, leading to limits in the polar
                 integration of 9.74$^\circ$ (vertical dashed violet line) and
                 7.76$^\circ$ (vertical dash-dotted violet line) respectively.
        }
        \label{fig:e_iso}
    \end{figure}

    \subsection{Monte-Carlo simulations}
    \label{sec:mc_sim}

    Using the information from the previous analyses and further priors motivated from
    them, a Monte-Carlo simulation was run where $10^5$ realisations of the Gaussian jet
    were made. Their model fluence was compared with what was reported by INTErnational
    Gamma-Ray Astrophysics Laboratory (INTEGRAL).\\ Around the time of the GW trigger,
    INTEGRAL was observing the entire AdvLIGO/VIRGO localisation region and according to
    \cite{minaev_gcn_2019}, saw a low SNR short duration ($\sim$ 1s) excess roughly 6s
    after the merger. Further analysis reported a fluence of $(1.6 \pm 0.4) \times
    10^{-7} \text{ erg/cm}^2$. The priors used for the various parameters of the
    Gaussian jets realised are as given in table \ref{table:priors}.

    \begin{table}[H]
        \centering
        \setlength{\extrarowheight}{0pt}
        \addtolength{\extrarowheight}{\aboverulesep}
        \addtolength{\extrarowheight}{\belowrulesep}
        \setlength{\aboverulesep}{0pt}
        \setlength{\belowrulesep}{0pt}
        \begin{tabular}{|c|c|c|c|}
            \toprule
            & $E_{tot., \gamma}$ (erg) & $\Gamma_0$ & $\theta_c$ \\
            \hline
            \rowcolor[rgb]{0.812,0.812,0.812}
            \textbf{Uniform Energy Prior} & $ 44 < \log_{10}(E_{tot., \gamma}) < 51$ &
            $5 < \Gamma_0 < 500$ & $3^{\circ} < \theta_c < 20^{\circ}$ \\
            \hline
            \begin{tabular}[c]{@{}c@{}}
                \textbf{Broken Power Law} \\
                \textbf{Energy Prior}
            \end{tabular} &
            \begin{tabular}[c]{@{}c@{}}
                BPL \{ [$5 \times 10^{47}, 10^{50}; -0.53$], \\{[}$10^{50}, 5 \times 10^{51}; -3.5$] \}
            \end{tabular} &
            $ 100 \Gamma_0 500$ & $3^{\circ} < \theta_c < 20^{\circ}$  \\
            \bottomrule
        \end{tabular}
        \caption[Priors on $E_{tot., \gamma}$, $\Gamma_0$ and $\theta_c$]{
            Priors on the total energy emitted in gamma-rays ($E_{tot., \gamma}$), bulk
            on-axis Lorentz factor ($\Gamma_0$) and core-angle ($\theta_c$). The
            notation used for the broken power-law distribution is explained below.
        }
        \label{table:priors}
    \end{table}

    Here, BPL$\{[5 \times 10^{47}, 10^{50}; -0.53], [10^{50}, 5 \times 10^{51}; -3.5]\}$
    is equivalent to:
    \begin{equation}
        \label{eq:7}
         P(E_{tot., \gamma}) \propto
        \begin{cases}
            E_{tot., \gamma}^{-0.53}, &
                5 \times 10^{47} \text{ ergs } <
                E_{tot., \gamma} <
                10^{50} \text{ ergs }  \\
            E_{tot., \gamma}^{-3.5}, &
                10^{50} \text{ ergs } <
                E_{tot., \gamma} <
                5 \times 10^{51} \text{ ergs }
        \end{cases}
    \end{equation}

    \begin{figure}[H]
        \centering
        \def\svgwidth{0.5\textwidth}
        \input{figures/bpl_demo.pdf_tex}
        \caption[Broken power law distribution from \cite{ghirlanda_2016}.]{
            The broken power law distribution as described in Eq. \ref{eq:7}.
        }
        \label{fig:bpl_demo}
    \end{figure}

    This particular distribution for the prior of $E_{tot., \gamma}$ is adopted since it
    is able to reproduce the fluence distribution observed, for values above the
    limiting fluence of $2 \times 10^{-7} \text{ erg/cm}^2$ (see \cite{mohan_2019}).
    Furthermore, the power-law indices are adopted from the luminosity function
    described in \cite{ghirlanda_2016}.

    In applying these priors, along with the $D_L-\iota$ prior and the fluence values of
    $2 \times 10^{-7}$ and $(1.6 \pm 0.4) \times 10^{-7}$ erg/cm$^2$ as upper limit and
    detections respectively, supplied by INTEGRAL, the marginalized posteriors for
    $\theta_c, \Gamma_0, \theta_v$ and $E_{tot., \gamma}$ are obtained. These are
    converted into $E_{iso}(0)$. The posterior distributions of the on-axis, apparent
    isotropic equivalent energy, for the two priors considered, is shown in Fig.
    \ref{fig:unif_bpl}. As is evident from seeing the figure, the INTEGRAL fluence is a
    good constraint for the priors. For the uniform prior case, considered as a
    detection, the posterior $E_{iso}(0)$ is tightly constrained to be between
    $3.51\times 10^{47} - 6.26 \times 10^{52}$ ergs, which shows that for an on-axis
    observer, the event would have appeared as a typical sGRB along with the GW event.
    Even considering as an upper-limit constrains $E_{iso}(0) \leq 1.48\times 10^{51}$,
    which is broadly in agreement with that observed for typical sGRBS.  On the other
    hand, the narrower, broken power-law prior is not constrained well with the INTEGRAL
    fluence. Considered as a detection, the 90\% credible posterior bounds on
    $E_{iso}(0)$ are $1.17\times10^{49}-1.3\times10^{51}$ ergs, whereas considered as an
    upper limit, $E_{iso}(0) \leq 7.69 \times 10^{50}$ erg. In both cases, the
    posteriors are sensitive to the choice of the prior, but nevertheless, one cannot
    rule out an sGRB jet which would have been seen by an on-axis observer.

    \begin{figure}[H]
        \begin{subfigure}{0.5\textwidth}
              \label{fig:unif}
              \centering
              \def\svgwidth{\textwidth}
              \input{figures/unif.pdf_tex}
        \end{subfigure}%
        \begin{subfigure}{0.5\textwidth}
              \label{fig:bpl}
              \centering
              \def\svgwidth{\textwidth}
              \input{figures/bpl_1.9.pdf_tex}
        \end{subfigure}
        \caption[Posterior distributions of the apparent on-axis isotropic equivalent
        energy $E_{iso}(0)$, for two assumed priors on the total energy emitted in
        gamma-rays, $E_{tot., \gamma}$]{
            Posterior distributions of the apparent on-axis isotropic equivalent energy
            $E_{iso}(0)$, for two assumed priors on the total energy emitted in
            gamma-rays, $E_{tot., \gamma}$. These figures give constraints on the
            $E_{iso}(0)$ of the sGRB associated with GW190425, assuming a Gaussian
            structured jet. \textbf{Left:} Grey histogram indicates the uniform priors
            on $\log_{10}(E_{tot., \gamma}/erg)$ in the range of [44 --- 51], and on
            $\theta_c$ in [3, 20] degrees.  \textbf{Right:} the same prior is used for
            $\theta_c$ but a broken power law prior is used for $E_{tot., \gamma}$. The
            orange histograms in both are a result of considering an INTEGRAL fluence
            upper limit of $2 \times 10^{-7} \text{ erg/cm}^2$, where the blue
            histograms in both are a result of considering an INTEGRAL fluence detection
            of $(1.6 \pm 0.4) \times 10^{-7} \text{ erg/cm}^2$. In both cases, the
            on-axis energy of a possible associated GRB is within the range of that of
            the cosmological sGRB population.
        }
        \label{fig:unif_bpl}
    \end{figure}

    %Write about the Monte-Carlo simulations undertaken to use the two pieces of
    %information from the two previous sections, to constrain the priors input, and see
    %whether the event S190425c could have possibly been a typical sGRB. Need to get
    %90\% credible intervals!

    \subsection{Using LIGO posteriors}

    In the time since the analysis in \cite{saleem_2020B} was carried out, the
    posteriors for the O3a events were released as part of the Gravitational Wave
    Transient Catalog (GWTC) 2 (see \cite{abbott_2020A}). These are now available as part
    of the event portal at the Gravitational Wave Open Science Center (GWOSC), which
    lists the files that store samples from posterior distributions for various
    parameters, for each event in GWTC-2. This allows one to use the actual posteriors
    for the inclination angle and the luminosity distance reported by the LVC. These are
    plotted in Fig. \ref{fig:dl-iota_post_updated}, below.\\

    \begin{figure}[H]
        \centering
        \def\svgwidth{\textwidth}
        \input{figures/dL-iota_updated.pdf_tex}
        \caption[$D_L-\iota$ posterior, with samples from the data released by LVC.]{
            Similar as \ref{fig:dl_iota}, but with samples for $D_L$ and $\iota$ from
            the posteriors released by the LVC, as part of the GWTC-2 data release.
        }
        \label{fig:dl-iota_post_updated}
    \end{figure}

    These posteriors for the parameters $\iota$ and $D_L$ will be more accurate than the
    ones generated in \S\S\ref{sec:dl-iota_posterior}, since those posteriors are
    generated by approximating the O3a detector noise curves using the conservative best
    estimate of O2. Hence using the actual detector noise around the time of the event
    and performing parameter estimation for the parameters of interest (which is done by
    LVC), and using the resultant posteriors will be more accurate for further
    analysis.\\
    Using these posteriors, the constraints on the energetics of a sGRB jet
    being powered by an event like GW190425 change slightly. This is shown in Fig.
    \ref{fig:unif_bpl_updated}.

    \begin{figure}[H]
        \begin{subfigure}{0.5\textwidth}
              \label{fig:unif_updated}
              \centering
              \def\svgwidth{\textwidth}
              \input{figures/unif_up.pdf_tex}
        \end{subfigure}%
        \begin{subfigure}{0.5\textwidth}
              \label{fig:bpl_updated}
              \centering
              \def\svgwidth{\textwidth}
              \input{figures/bpl_1.9_up.pdf_tex}
        \end{subfigure}
        \caption[Posterior distributions of the apparent on-axis isotropic equivalent
        energy $E_{iso}(0)$, with similar priors as Fig. \ref{fig:unif_bpl} but with LVC
        posterior samples as input.]{
            Posterior distributions of the apparent on-axis isotropic equivalent energy
            $E_{iso}(0)$, with similar priors as Fig.  \ref{fig:unif_bpl} but with LVC
            posterior samples as input.
        }
        \label{fig:unif_bpl_updated}
    \end{figure}

    Considering the INTEGRAL fluence as a detection, the posterior bounds on
    $E_{iso}(0)$ are $5.61 \times 10^{47} - 8.48 \times 10^{52}$ ergs. This again leads
    to the conclusion that for an on-axis observer, the sGRB jet would have been
    detectable. Considering the INTEGRAL fluence as an upper limit instead, givens that
    $E_{iso}(0) \leq 7.43 \times 10^{51}$ erg.  Similar to the previous case, the narrow
    broken-power law prior is not well constrained by the INTEGRAL fluence limits, which
    places the bounds $1.15 \times 10^{49}-1.11 \times 10^{51}$ ergs (considered as a
    detection) and $E_{iso}(0) \leq 6.74 \times 10^{50}$ erg (considered as an upper
    limit). In this case as well, the conclusion is the same, that the possibility of
    sGRB which would have been visible seen on-axis cannot be ruled out. However, this
    showcases the usefulness of the method described in \S \ref{sec:dl-iota_posterior},
    wherein the posterior for the parameter $\iota$ can be approximated and used for
    further analysis, without having to wait for this information to be released
    officially.

\section{Preliminary analysis of GW190426\_152155}\label{sec:190426}

    %Talk about the preliminary results of using the GW190426 posteriors, and how that
    %is not constraining enough. Also put comparison with the other "typical" GRBs.

    The event GW190426\_152155 is listed in GWTC-2 (\cite{abbott_2020A}) as an event with
    a network matched filter signal-to-noise ratio of 10.1, a false-alarm rate of 1
    event per 1.6276 yr and the component masses are $5.7^{+4.0}_{-2.3}$ $M_{\odot}$ and
    $1.5^{+0.8}_{-0.5}$ $M_{\odot}$. From GRACE-DB, this event has probabilities 0.58,
    0.24, 0.12, 0.06 respectively of being a Terrestrial, BNS merger, NSBH merger or
    MassGap merger event.\\ Although there were no significant excesses reported by any
    of the Gamma-ray satellites observing the LVC localisation area at the time of the
    GW trigger, the INTEGRAL satellite reported an upper limit fluence of $1.7 \times
    10^{-7} \text{ ergs/cm}^2$. With the priors as given in Table \ref{table:priors},
    this fluence upper limit reported by INTEGRAL is taken as the primary constraint.
    Similar to the process described in \S\ref{sec:mc_sim}, $10^5$ realisations of the
    jet are made, the apparent on-axis isotropic equivalent energy calculated, from
    which the fluence is computed and those realisations with a fluence beyond the upper
    limit are rejected. The resulting population has a distribution as given in Fig.
    \ref{fig:nsbh_unif_bpl}.\\
    As can be seen from Fig.\ref{fig:nsbh_unif_bpl}, at apparent on-axis isotropic
    equivalent energies below $10^{48}$ erg, the uniform prior differs appreciably from
    the posterior, whereas at higher energies the posterior and prior exhibit the same
    behaviour. In the case of the broken power-law prior, at all energies considered,
    the posterior and prior distributions are  the fluence upper limit offered by
    INTEGRAl doesn't offer tight constraints. This is because for both the priors, the
    posterior largely follows the same behaviour as the prior, thus indicating that the
    constraint applied is a poor one.

    \begin{figure}[H]
        \begin{subfigure}{0.5\textwidth}
              \label{fig:nsbh_unif}
              \centering
              \def\svgwidth{\textwidth}
              \input{figures/unif_nsbh.pdf_tex}
        \end{subfigure}%
        \begin{subfigure}{0.5\textwidth}
              \label{fig:nsbh_bpl}
              \centering
              \def\svgwidth{\textwidth}
              \input{figures/bpl_1.9_nsbh.pdf_tex}
        \end{subfigure}
        \caption[Posterior distributions of the apparent on-axis isotropic equivalent
        energy $E_{iso}(0)$, in the case of GW190426\_152155.]{
            Posterior distributions of the apparent on-axis isotropic equivalent energy
            $E_{iso}(0)$, for two assumed priors on the total energy emitted in
            gamma-rays, $E_{tot., \gamma}$. These figures give constraints on the
            $E_{iso}(0)$ of the sGRB associated with GW190426\_152155, assuming a
            Gaussian structured jet. \textbf{Left:} Grey histogram indicates the uniform
            priors on $\log_{10}(E_{tot., \gamma}/erg)$ in the range of [44 --- 51], and
            on $\theta_c$ in [3, 20] degrees. \textbf{Right:} the same prior is used for
            $\theta_c$ but a broken power law prior is used for $E_{tot., \gamma}$. The
            orange histograms in both are a result of considering an INTEGRAL fluence
            upper limit of $1.7 \times 10^{-7} \text{ erg/cm}^2$.
        }
        \label{fig:nsbh_unif_bpl}
    \end{figure}

    This process sees whether the models used for the analysis of GW190425 also apply to
    the NSBH event GW190426\_152155. Although the constraint is not very good in either
    case of the prior, the conclusion that there is the possibility of a sGRB jet which
    could have been detected had the observer been on-axis, cannot be ruled out
    completely. In this case, considering the constraint as an upper limit, the bounds
    are $E_{iso}(0) \leq 4.52 \times 10^{51}$ erg and $E_{iso}(0) \leq 6.81 \times
    10^{50}$ erg for the uniform prior and the broken power-law prior respectively.  As
    an additional check, table \ref{table:typical_grbs} lists the parameters for some of
    the typical, well-studied sGRBs. Comparing what was obtained after the analysis of
    GW190426\_152155 to the observed isotropic equivalent energy for these sGRBs, it can
    be seen that this NSBH event is still in the ballpark of cosmological sGRBs.

    \begin{table}
        \centering
        \setlength{\extrarowheight}{0pt}
        \addtolength{\extrarowheight}{\aboverulesep}
        \addtolength{\extrarowheight}{\belowrulesep}
        \setlength{\aboverulesep}{0pt}
        \setlength{\belowrulesep}{0pt}
        \begin{tabular}{ccccccc}
            \toprule
            \rowcolor[rgb]{1,0.741,0}
            \textbf{GRB ID} &
            \begin{tabular}[c]{@{}>{\cellcolor[rgb]{1,0.741,0}}c@{}}
                \textbf{Relevant} \\ \textbf{ GCN} \\ \textbf{ Notices}
            \end{tabular} &
            \begin{tabular}[c]{@{}>{\cellcolor[rgb]{1,0.741,0}}c@{}}
                \textbf{Duration} \\ \textbf{(ms)}
            \end{tabular} &
            \begin{tabular}[c]{@{}>{\cellcolor[rgb]{1,0.741,0}}c@{}}
                \textbf{Fluence} \\ \textbf{($\text{erg/cm}^2$)}
            \end{tabular} &
            \textbf{Redshift} &
            \begin{tabular}[c]{@{}>{\cellcolor[rgb]{1,0.741,0}}c@{}}
                \textbf{Lum. Dist} \\ \textbf{(Mpc)}
            \end{tabular} &
            \begin{tabular}[c]{@{}>{\cellcolor[rgb]{1,0.741,0}}c@{}}
                \textbf{$E_{iso}$} \\ \textbf{(ergs)}
            \end{tabular}  \\
            GRB20050509B &
            \begin{tabular}[c]{@{}c@{}}
                3385, \\ 3390
            \end{tabular} &
            30 & $2.3^{+0.9}_{-0.9} \times 10^{-8}$ & 0.226 & 1133.2 &
            $3.53^{+1.38}_{-1.38} \times 10^{48}$ \\
            GRB20130603B &
            \begin{tabular}[c]{@{}c@{}}
                14741, \\ 14744
            \end{tabular} &
            180 & $(6.3^{+0.3}_{-0.3}) \times 10^{-7}$ & 0.356 & 1911.9 &
            $2.76^{+0.044}_{-0.044}\times10^{50}$ \\
            GRB20160821B &
            \begin{tabular}[c]{@{}c@{}}
                19844, \\ 19846
            \end{tabular} &
            480 & $(1.0^{+0.1}_{-0.1}) \times 10^{-7}$ & 0.162 & 781.8 &
            $7.31^{+0.73}_{-0.73}\times 10^{48}$ \\
            \bottomrule
        \end{tabular}
        \caption[List of typical cosmological sGRBs and their source parameters]{
            List of typical cosmological sGRBs and their source parameters. The
            luminosity distance was calculated from the redshift using a standard
            cosmology of $H_0 = 69.6 \text{ km/s/Mpc}$, $\Omega_M = 0.286$ and
            $\Omega_\Lambda = 0.714$.
        }
        \label{table:typical_grbs}
    \end{table}

    %Also, put in Fisher matrix approach if possible.

\section{Summary}
